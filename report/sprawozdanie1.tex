\documentclass[a4paper,11pt]{article}
\usepackage[T1]{fontenc}
\usepackage[polish]{babel}
\usepackage[utf8]{inputenc}
\usepackage{lmodern}
\selectlanguage{polish}
\title{
	\textbf{Programowanie równoległe i rozproszone}\vspace{40pt}
	\textit{Politechnika Krakowska} \\\vspace{40pt}
	Laboratorium 1
	\vspace{300pt}

}
\author{
	Paweł Suchanicz,\\
	Rafał Niemczyk
}
\begin{document}
\begin{titlepage}
\maketitle
\end{titlepage}

\begin{center}
\tableofcontents
\end{center}
\newpage
\section{Wstęp}
\subsection{Opis laboratorium}
\paragraph{}Celem laboratorium było wykorzystanie interfejsu OpenMP w celu zrównoleglenia kodu C++. Interfejs OpenMP składa się głównie z dyrektyw preprocesora a także z zmiennych środowiskowych i funkcji bibliotecznych. W laboratorium wykorzystywany będzie głównie do zrównoleglania pętli.
\paragraph{}Algorytmy, które są implementowane a następnie zrównoleglane w ramach laboratorium to normalizacja min-max, standaryzacja rozkładem normalnym i klasyfikacja KNN (k-najbliższych sąsiadów). Zaimplementowany KNN  uwzględnia jednego sąsiada i używa metryki euklidesowej.
\paragraph{}Szybkość działania każdego algorytmu została zmierzona dla implementacji w C++, implementacji w C++ po zrównolegleniu dla różnej ilości wątków (1-4) oraz impelmentacji w Python (ze skorzystaniem z funkcji z pakietu scikit-learn).
\subsection{Specyfikacja sprzętowa}
\paragraph{}Przy pomiarach szybkości wykonywania algorytmów wykorzystany był sprzęt w konfiguracji:
\begin{itemize}
\item Procesor: Intel Core i7-4712MQ 4 x 2.30GHz
\item Ram: 8GB DDR3
\item System: Linux (Fedora 22)
\end{itemize}
\subsection{Zbiór danych} 
\paragraph{}Wykorzytany został zbiór obrazów ręcznie pisanych cyfr MNIST. Wykorzytany zbiór ma format .csv i zawiera 60000 rekordów, gdzie każdy rekord odpowiada za jeden obrazek 28x28 pikseli w skali szarości. Pierwsza wartość w rekordzie jest cyfrą która widnieje na obrazku, a kolejne to wartości pikseli obrazka. 
\paragraph{}
Dla zadań postawionych w laboratorium zbiór danych jest dość duży, więc został on obcięty do pierwszych 6000 rekordów, z czego 4500 przeznaczono do trenowania, a pozostałe 1500 do testowania.
\newpage    
\section{Wyniki}   
\subsection{Normalizacja i standaryzacja} 
\paragraph{}Normalizacja:
\paragraph{}$x^*=\frac{x-min(x)}{max(x)-min(x)}$
\paragraph{} Standaryzacja:
\paragraph{}$x^*=\frac{x-\mu}{\sigma}$
\subsubsection{Implementacja w C++} 
\subsubsection{Implementacja w Python} 
\subsubsection{Porównanie wyników} 
\newpage
\subsection{Klasyfikacja KNN} 
\subsubsection{Implementacja w C++} 
\subsubsection{Implementacja w Python} 
\subsubsection{Porównanie wyników} 
\newpage
\section{Podsumowanie} 
 
\end{document}